\chapter{Abstract}
Cloud computing and container technology have experienced widespread interest for the last couple of years. Academia, however, is challenged in providing courses and material to embrace this new paradigm. Currently, most research into cloud computing has evolved around the use of software simulation, expensive data centers, and large setups of low-cost Raspberry Pi clusters unable to be handed out to students. We introduce KubeCloud, a small-scale tangible cloud computing environment. KubeCloud allows students to experiment with cloud computing concepts and theories with a hands-on, practical learning object. KubeCloud is a controlled test environment that can be used by researchers to experiment with a low-cost, small-scale model of a cloud computing environment. KubeCloud consists of four Raspberry Pi 2 model B and allows for easy extension of multiple KubeClouds to form a larger scale-model. In this thesis, we present the design of KubeCloud and a designed learning activity, the execution of the learning activity with KubeCloud, and an evaluation of KubeCloud as a learning and research object.
The evaluations suggest that KubeCloud can help students break down the abstractions of cloud computing and make them visible in a physical cluster. \\ \\

\noindent
\textbf{Keywords:} Cloud Computing, Microservices, Kubernetes, Docker, Containers, Teaching strategies, Learning Objects, Resilience