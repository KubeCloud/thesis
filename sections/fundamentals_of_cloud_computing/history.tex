%!TEX root = ../../master.tex
\section{History of Cloud computing}
How we build software evolves over time, new methods, languages, technologies, etc. emerge all the time. Some are being adopted by the masses and some involves into standards and best practices for particular problems. This of course also happens within the field of cloud computing. The idea of an "intergalactic computer network" was introduced in the sixties by J.C.R. Licklider, who was responsible for enabling the development of ARPANET (Advanced Research Projects Agency Network) in 1969\cite{history_of_cloud_computing}. Since then one of the first important milestones within cloud computing happened in 1999 when Salesforce.com launched the concept of delivering enterprise applications via a simple website. In 2002 Amazon joined the cloud computing business and offered a suite of cloud-based services including storage and computation. In 2006 Amazon launched its Elastic Compute cloud (EC2), a commercial web service that provides small companies and individuals the possibility to rent computeres on which they can run their applications. In 2009 Google and others joined cloud computing and together with the evolution and maturing of virtualization technology, the development of universal high-speed bandwith, and universal software interoperability standards the era of cloud computing was born. Since 2009 the emergence of "as a Service" paradigm which introduced numerous different layers of applying services in the cloud. Infrastructure-as-a-Service (IaaS), Platform-as-a-Service (PaaS) and Software-as-a-Service are all terms for different services provided by different cloud vendors. In 2013 the initial release of Docker, helped spark the interest even further. Docker introduced a standard for packaging and application and its dependencies in a virtual container that can run on any Linux server. This enables the flexibility and portability on where applications can run, whether its on premise, in the public cloud, the private cloud or bare metal.

