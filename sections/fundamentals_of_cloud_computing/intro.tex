%!TEX root = ../../master.tex
\begin{theorem}
    \textit{An understanding of cloud computing infrastructure fundamentals is necessary to design and build a learning activity and a tangible cloud computing cluster.}
\end{theorem}

\noindent
Until now, we have stressed the importance of learning theory and applied it to design a learning activity for engineering students. The focus of the next three chapters (including this chapter) will change direction by diving into the details of the course content and the necessary knowledge to design and implement a tangible cloud computing cluster, KubeCloud. This chapter will provide a definition of the enabling technologies in the cloud computing stack and infrastructure. The next chapter (Chapter~\ref{chap:fundamentals_cloud_computing_architecture}) will describe the fundamentals of cloud architecture and how to design applications in a service-oriented manner. Lastly, the implications of applying a service-oriented architecture will be discussed in Chapter~\ref{chap_fundamentals_resilient_cloud} in regards to faults and resilience.