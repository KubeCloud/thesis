%!TEX root = ../../master.tex
\section{Crash-only software (2003)}
Stopping crash-only software == crashing \\
Starting crash-only software == recovering \\
Intra-component crashes -> retries. Transparent from end user.\\
Crash-only => More reliable, predictable code + faster effective recovery \\
\textbf{1) Occam's Razor} \\
Sources of downtime (referencing paper) is caused by transient and intermittent bugs  \\ \\
Unavailability comparison: \\
Cleanly-reboot: shutdown + "time to come back" \\
Crash-reboot: "time to recover" \\
Crash-reboot is faster, but can lead to inconsistent state \\
\\
It is impractical to guarantee no crashes - Power off is always an option \\
The software must be able to handle crash (shutdown) \\
If crash-prepared, why support more kinds of shutdown? (clean shutdown) \\
Inconsistent state can be a problem \\
Clean shutdown prioritizes: "stady state" > shutdown/recovery performance \\
Clean shutdown introduces an extra way to change state \\
Harder to maintain \\
\\
Other articles about "micro-reboots" and "recursive restart" \\
\\
\textbf{STOP=CRASH} \\
\textbf{START=RECOVER} \\

\textbf{2) Why crash-only design?} \\
In physics: Descriptive physical laws (like Newton's law) \\
In software: Prescriptive rules (formal methods, invariant proofs) formulated relative to an ABSTRACT MODEL that does not completely describe the behaviour of running system. It ignores: hardware, OS, runtime libraries etc. \\
\\
Macroscoping behaviour. Simpler predictable "universe". \\
Components MUST be ready for crashes \\
Makes crash the default instead of something seldom happening (hard to test) \\
Crash-only design is a generalization of transactions (use data-store) \\
Makes it affordable to crash. \\
Unkonwn fault => crash => recover => available again \\

\textbf{3: Properties (what)} \\
Important non-volatile state in dedicated "state stores" (allows crashes) \\
Components tolerate crash/unavailability of peers (modularity, boundaries, timeouts, self-describing requests, time-to-live)\\
Sacrifice some performance for robustness \\
\\
3.1: Intra-component properties \\
different kinds of state \\
Important non-volatile => dedicated "state stores" => stateless apps \\
"state stores" must be crash-only, e.g. Postgres "append-only log" allows discard if uncommitted. \\
Find the suitable store per component \\
\\
3.2: Inter-component properties \\
Subsystems can crash => unavailability \\
Externally enforaced boundaries (VMs or alike) \\
Use timeouts and assume failures => recovery agent can crash component \\
Leases (for resources) with expiration \\
Self-describing requests: URL, TTL, idempotent \\
\\
\textbf{4) A restart/retry Architecture}
Component: \\
\begin{itemize}
	\item report failure of peer to agent
	\item calls RetryAfter(n) n is msec estimate of time-to-recover
	\item if TTL isn't OK, propagate exception
\end{itemize}
Timeouts supplemented with heartbeats and progress counters \\
Loosely coupled systems are surprising \\
Resubmitting req => overload => solution: spread retry wait value \\
Use "stall proxy" to avoud flood during recovery -> introduce latency \\

\textbf{5) Discussion} \\
Applying principles in OpenSource webserver (referenced article?) \\
Not handling Byzantine failure and data errors \\
Fast recovery => less redundancy needed \\
Lower throughput expected but higher availability \\

\textbf{6) Conclusion} \\
They are EXPECTING better reachability, higher availability, and simpler fault models. \\
Invariants easier to predict \\
Initial prototype 78\% more requests under "faultload"