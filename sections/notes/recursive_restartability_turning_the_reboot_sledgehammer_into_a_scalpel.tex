%!TEX root = ../../master.tex
\section{Recursive Restartability: Turning the Reboot Sledgehammer into a Scalpel, Candea, Fox, 2001}

\begin{itemize}
	\item Even after decades of software engineering research, complex computer systems still fail, primarily due to nondeterministic bugs that are typically resolved by rebooting.
	\item Heisenbugs will remain a fact of life
	\item \textbf{Recursive Restartability:} the ability of a system to gracefully tolerate restarts at multiple levels - improves fault tolerance, reduces time-to-repair, and enables system designers to build flexible, highly available software infrastructures. 
	\item The rebooting "technique" has been around as long as computers themselves, and remains a fact of life for substantially all nontrivial systems today.
	\item Deadlock resolution in commercial database systems is typically implemented by killing and restarting a deadlocked thread in hopes of avoiding a repeat deadlock. 
	\item A major search engine periodically performs rolling reboots of all nodes in their search engine cluster.
	\item Rebooting is often only a crude "sledgehammer" for maintaining system availability, its use is motivated by two common properties:
	\begin{itemize}
		\item Restarting works around Heisenbugs
		\item Restarting can reclaim stale resources and clean up corrupt state
	\end{itemize}
	\item We argue that in an appropriately designed system, we can improve overall system availability through a combination of reactively restarting failed components (revival) and prophylactically restarting functioning components (rejuvenation) to prevent state degradation that may lead to unscheduled downtime.
	\item An alternate definition of RR is provided by the recursive contruction: the base case RR system is a restartable software component: a general RR system is a composite of RR systems that obeys the guidelines of section 4. 
	\item Properties of RR
	\begin{itemize}
		\item RR improves fault tolerance
		\item RR can make restarts cheap
		\item RR provides a confidence continuum for restarts
		\item RR enables flexible availability tradeoffs
	\end{itemize}
	\item "The overarching theme is that of designing applications as loosely coupled distributed systems, even if they are not distributed in nature."
	\item \textbf{Accepting no for an answer:} Software components should be designed such that they can deny service for any request or call.
	\item \textbf{Using reconstructable soft state with announce/listen protocols: } 
	\begin{itemize}
  		\item Announce/listen makes the default assumption that a component is unavailable unless it says otherwise
  		\item Soft state can provide information that will carry a system through a transient failure of the authoritative data source for that state.
	\end{itemize}
	\item \textbf{Automatically trading precision or consistency for availability: }
	\begin{itemize}
		\item Inter-component "glue" protocols should allow components to make dynamic decisions on trading consistency/precision for availability, based on both application specific consistency/precision measures, and a consistency/precision utility function.
	\end{itemize}
	\item \textbf{Structuring applications around fine grain workloads}
	\begin{itemize}
		\item "Glue" protocols should enforce fine grain interactions between subsystems. They should provide hooks for computing the cost of a subsystem's restart based on the expected duration of its current task and its children's tasks.
	\end{itemize}
	\item \textbf{Using orthogonal composition axes: }
	\begin{itemize}
		\item Independt subsystems that do not require an understanding of each other's functionality are said to be mutually orthogonal.
		\item Split functionality along orthogonal axes. Each corresponding subsystem should be centered around an independent locus of control, and interact with other subsystems via events posted using an asynchronous mechanism.
	\end{itemize}
	\item Availability benchmarking has been of interest only for the pas decade.
	\item "A fast system that is hung or crashed is simply an infinitely slow system"
	\item \textbf{Conclusion: } We proposed a turning the reboot from a demonic concept into a reliable partner in the fight against system downtime, given that it is a time-tested, effective technique for circumventing Heisenbugs.

\end{itemize}
