%!TEX root = ../../master.tex
\section{The Resilience of Networked Infrastructure Systems (Chapter 2)}
\textbf{Resilience Definitions}
\begin{itemize}
  \item Latin word resilíre: "to leap back." ~ "bounce back"
  \item Folke et al. (2002): the capacity of the system to absorb disturbance and reorganize while undergoing change so as to retain the same function, structure, identity and feedbacks.
  \item Dalizell and McManus (2004): the overarching goal of a system to continue to function to the fullest possible extend in the face of stress to achieve its purpose.
  \item Rose and Liao (2005): the ability of the system to apply adaptive responses in the face of disruptions in order to avoid potential losses.
  \item Andersson (2006): the ability of an actor ... to cope with or adapt to hazard stress... Resilience is influenced by proactive/reactive measures
  \item Fiskel (2003): ability to return to a stable equilibrium state after perturbation
  \item Bruneau et al. (2003): (system) that has reduced failure probability, reduced consequences of failures, and reduced time to recovery... robustness, redundancy, resourcefulness and rapidity to be properties of resilient systems
  \item Reed et al. (2009): quality as the capacity of the infrastructure... robustness to be the ratio of the lost capacity of the system as a result of a disruptive event to the capacity of a fully functioning structural system... rapidity to be the measure of rapidity of recovery.
  \item Pavard et al. (2006): capable of maintaining a constant output value level when the system suffers from a perturbation
  \item Wears and Perry (2008): A resilient system is able to adapt to the shock and contain it; a brittle system lacks the ability to adapt and transmits exogenous shocks.
\end{itemize}

\noindent
perturbation = uro, bestyrtelse \\
brittleness/fragility = skrøbelighed \\
equilibrium (points) = balance/ligevægt
\\
\\
2.2 Resilience in Different Disciplines \\
e.g. ecological systems, psychology, business organizations...
\\ \\
2.3 Resilience and Disruptions (Shocks) \\
main factors that cause disruptions
\\
2.3.1 Categories of potential disruption \\
Factors by Manrouri et al. (2009b):
\begin{itemize}
  \item Human - malicious attacks or accidental disruptions/error
  \item Natural - nature, hurricanes and floods
  \item Organizational - e.g. workers strike
  \item Technical - faulty equipment 
\end{itemize}

\noindent
\\
regular threats are predictable \\
irregular threats are difficult to predict \\
\\
2.3.2 Disruption profile \\
Sheffi: Typically have 8 phases (see figure on p. 17) \\
"The purspose of implementing resilience is to change the appearance of the disruptive profile by reducing the area between the dotted line representing the normal performance of the system and the dip, illustrated by the solid line, reflecting the impact of the disruptive event on the performance." \\
\\
2.4 Methodologies for Characterizing Resilience \\
Bruneau and Reinhorn (2007) \\
\begin{itemize}
  \item Robustness - withstand disruption
  \item Redundancy - substitutes functioning after a disruptive event
  \item Resourcefulness - capacity to identity, prioritize and apply resources
  \item Rapidity - capacity to meet priorities
\end{itemize}

\noindent
\\
Fiksel (2003): Having \textit{multiple equilibrium points} leads to resilience
\\
\\
Walker et al (2004):
\begin{itemize}
  \item Latitude - amount of change before system loses its ability to recover
  \item Resistance - ease of changing
  \item Precariousness - how close to system threshold
\end{itemize}

\noindent
\\
Richards et al. (2007): Two phases of resilience. 1) anticipation/avoidance phase 2) survival/recovery phase. \\
\\
2.5 Resilience Measurement Approaches
\\
Bruneau and Reinhord (2007): \textbf{resilience triangle}
resilience can be measured as the time to recovery (formula on p. 21)
\\
\\
2.6 Elements of Resilience
\\
Sheffi (2005): the system's \textbf{vulnerability} as the combination of the likelihood of a disruption and the potential severity of the disruption.
Figure showing \textbf{disruption probability vs. consequences}.
\\
\\
Dalziell and MacManus (2004): define a system of high vulnerability as a system that may be easily pushed from one state of stability or equilibrium to another, where as it is not easy to push a system with low vulnerability from a state of stability or equilibrium.
\\
\\
Dalziell and MacManus (2004) refer to resilience as being a function of both the vulnerability of the system and its adaptive capacity. The relationship between resilience and vulnerability is bidirectional: \textbf{loss of resilience will result in the increasing vulnerability} of the system and would \textbf{increase the risk of shifting the system into an undesirable state}.
\\
adaptive capacity: "the capacity to adapt and to reconfigure in the face of disruptions without losing functionality" \\
\\
Folke et al: cope with novel situations \\
\\
2.8 Resilience and Risk Management \\
Risk management deals with the \textbf{identification of threats and vulnerabilities} to determine risk and to apply mitigation strategies that reduce or avoid that risk.
\\ \\
Risk management procedures are used for the identification of schemes that enhance resilience based on the vulnerabilities of the systems and their potential impact.
\\
\\
Achieving resilience in systems extends beyond making the system less vulnerable to disruptions; it also entails the implementation of adaptive measures that will enable the system to resume functionality with minimum losses if a threat does impact the system.


\section{Relationship Between Reliability Robustness, Flexibility, Agility And Resilience Chapter 3}
\begin{itemize}
  \item \textbf{Reliability}: the ability of a system to function in a satisfactory manner in its predicted lifetime under stated conditions. Mathematical likelihood of failure-free performance over a  period of time. \\
  \begin{itemize}
    \item Reliability index: Probability between 0-1
    \item Mean Time to Failure (MTTF): \begin{equation}
    	MTTF=\frac{Total\ operating\ time\ for\ all\ components}{No.\ of\ failures\ over\ that\ time}
             \end{equation}
    \item Mean Time to Repair (MTTR): average time a piece of equipment performs its function without requiring repair
    \item Series-parallel systems, fault trees
    \item Reliability vs. resilience
         \begin{itemize}
            \item Designed for known failure circumstances vs. designed for unforeseen disruptive events
            \item Internal failures vs external failures
            \item System unable to reconfigure to avoid failure vs. System able to reconfigure to continue operation
         \end{itemize}
  \end{itemize} 
\end{itemize}
