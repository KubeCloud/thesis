%!TEX root = ../../master.tex
\section{Chapter 1: Fault Tolerance and Resilience: Meanings, Measures and Assessment, Stringini}

\begin{itemize}
  \item The word "resilience", from the Latin verb \textit{resilire} (to jump back), means literally the tendency or ability to spring back, and thus the ability of a body to recover its normal size and shape after being pushed or pulled out of shape, and therefore figuratively any ability to recover to normality after a disturbance.
  \item Concepts related to "resiliency" are fault-tolerance, redundancy, stability, feedback control.
  \item A useful meaning to apply to "resilience" for current and future ICT is "ability to deliver, maintain, improve service when facing threats and evolutionary changes".
  \item The important extension to emphasize in comparison with word like "fault tolerance" was the fact that the perturbations that current and future systems have to tolerate include change.
\end{itemize}

