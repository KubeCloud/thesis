%!TEX root = ../../master.tex
\begin{theorem}
Designing a learning activity is necessary to investigate the effects of KubeCloud. Learning theories must be applied within the constraints of the course format.
\end{theorem}

\noindent
Chapter \ref{chap_fundamentals_learning} covered some of the fundamentals of learning theory. In this chapter, we will apply these theories and concepts to design a learning activity focused on the use of a tangible cloud computing cluster. We propose a design of a learning activity using a tangible cloud computing cluster as a learning object. The focus of the learning activity is to motivate and engage students in practical problem solving in cloud computing. The following chapters will give a detailed explanation of the concepts and theories the designed learning activity is based on. \\

\noindent 
The designed learning activity has been taught in a course, Object-Oriented Network Communication\footnote{\url{http://kursuskatalog.au.dk/en/course/63876}} at Aarhus University School of Engineering,  to verify the learning outcome of the learning activity and the impact of the tangible cloud computing cluster. These results will be presented in Chapter~\ref{chapter_experiment2_resilience}, \ref{chapter_experiment1_learning_experience}.