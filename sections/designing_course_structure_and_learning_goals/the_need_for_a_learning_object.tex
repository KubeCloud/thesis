%!TEX root = ../../master.tex
\section{The Need for a Learning Object}
Activity-based theories describe the use of learning objects and the importance such an object can have in a learning experience. We see a need for designing a learning object that can help students understand the context from which many abstractions in cloud computing have been made, and propose \textit{KubeCloud - A Small-Scale Tangible Cloud Computing Environment}.\\

\noindent
Abstractions are one of the key concepts in cloud computing, and these abstractions have made it easy for developers to provision and deploy applications in the cloud. Failing to understand the premise of these abstractions can have great consequences. The paradigm shift from monolithic applications (using intra-process method calls) to multiple small services (using inter-process communication over a network) introduces a lot of complexity that has to be dealt with. The fallacies of distributed computing have to be addressed and made visible for students. KubeCloud will help students learn about the fallacies of distributed programming and provide a platform for experimentation and exploration in a controlled environment. An active process, in which the learner uses sensory input and constructs meaning out of it, is important (Chapter~\ref{chap_fundamentals_learning}). This view is also pointed out by Satterthwait, \textit{"Some of the most productive, and common, science activities are those that involve the manipulation of objects. This factor plays a significant role in motivating and focusing our students on the learning of Science through the use of objects in an activity in which they can be engaged"} \cite[p. 8]{satterthwait2010hands}. The need for hands-on experience was found in a survey conducted at Jordan University of Science and Technology during an investigation into the challenges of teaching Cloud Computing. 93\% of the students answered that there was a lack of hands-on experience \cite[p. 2]{jararweh2013teachcloud}. \\

\noindent
Chapter~\ref{chap:tangible_cluster} covers the details of the design of the KubeCloud. The course presented in this master's thesis will be designed around an active, student-based approach and the use of the tangible cloud computing cluster as an object for learning.