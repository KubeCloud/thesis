%!TEX root = ../../master.tex

\section{Load and Stress Testing}
When applications are developed and deployed in KubeCloud, we need to simulate user traffic on the cluster to apply real use cases. KubeCloud then becomes a controlled test environment for cloud computing research. In order to simulate traffic and experiment with how applications respond to different traffic loads, different attacks can be made. There are two different ways of attacking the application: \textit{load} and \textit{stress testing}. A load test is \textit{"a test focused on determining or validating performance characteristics of the software under test when subjected to workload models and load volumes anticipated during production operations"} \cite{techtarget2016testing}. In a stress test the subject to workload is pushed beyond the anticipated production operations.
\noindent Both types of tests have been performed in this thesis. In order to perform load and stress testing Vegata\footnote{\url{https://github.com/tsenart/vegeta}} and Gatling.io\footnote{\url{http://gatling.io/}} have been used. Vegeta allows for a constant request rate against a target whereas Gatling records usage scenarios and replay them.