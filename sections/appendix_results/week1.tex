%!TEX root = ../../master.tex
\noindent
\subsubsection*{Week 1: Reflection}

%%%% Q1
\renewcommand*{\arraystretch}{1.6}
\scriptsize
\begin{longtable}{|p{0.3cm}|p{14.7cm}|} 
\hline
\rowcolor[HTML]{EFEFEF} & \textbf{Question 1: What challenges do you see in cloud computing and microservice architectures?}  \\
\hline
\endfirsthead
\multicolumn{2}{c}%
{\tablename\ \thetable\ -- \textit{Continued from previous page}} \\
\hline
\rowcolor[HTML]{EFEFEF} &\textbf{Question 1: What challenges do you see in cloud computing and microservice architectures?}  \\
\hline
\endhead
\hline \multicolumn{2}{r}{\textit{Continued on next page}} \\
\caption{Question 1: What challenges do you see in cloud computing and microservice architectures?}
\endfoot
\caption{Question 1: What challenges do you see in cloud computing and microservice architectures?}
\label{w1_q1}
\endlastfoot

1 & Added complexity due to communication between different microservices
    \\ \hline
    
2 & One of the challenges in cloud computing, is knowing when to use it, and when it is not beneficial. A challenge of microservices, is to know how to divide a sevice into microservices. \\ \hline

3 & Keeping the overall complexity low and manage the services in a clever way. 
	
\noindent The IP addresses of the exposed services must be broad casted to consumers or handled by a gateway to simplify the configuration. \\ \hline

4 & Connections between devices
	
\noindent Being consistent with interfaces \\ \hline

5 & network communication. \\ \hline

6 & Udfordringerne er ikke mærkant anderledes end ved normal server setup, men er en evolution/udvikling herfra. Løsningen, så at sige. \\ \hline

7 & Load balancing: A specific node should not bee loaded more than the other

\noindent Availability: Data should be available in any time and during any circumstances. Data should for instance not be missed if a single node fails. \\ \hline

8 & Consistency between the interfaces of the different services, might pose a challenge, should the interface of one instance change it might break everything.

\noindent As for cloud computing, network communication and package loss might prove troublesome.\\ \hline

9 & good architecture is required for it to make sense.\\ \hline

10 & The major challanges I see in working with cloud computing and especially microservices is how to structurize them and how to let the services communicate internally. In a traditional monolithic application I believe it's easier to do that part but when the structure is right I'm sure that the microservices will be much easier to maintain since they aren't tightly coupled with other components which is often a problem in monolithic applications.\\ \hline

11 & Initial complexity might be a problem (creating a cloud/microservice application for a small task, that has only one responsibility, might be troublesome). If there's not a lot to process or handle in a system, a monolithic approach might be the easiest way to go, but if you want to be able to expand on this system at any given later time, distributed systems are obviously the way to go. This was also proven by the example given with the pi's.

\noindent Another issue is the bounded context, how to handle the inevitable problem of polysemes in bigger systems, and other boundary related issues.\\ \hline

12 & When connections are unstable and bad cloud computing has some challenges. Furthermore it is important to break down every system and architecture to small parts, so you can detect where the errors are located.\\ \hline

13 & The overhead to implement a cloud computing system is rather large - meaning that the system will need to have a decent size befrore the gain is larger than the expence. \\ \hline

\multicolumn{2}{r}{\textbf{Evaluation:} Unistructural} \\ 
\end{longtable}
\normalsize



%%%% Q2
\renewcommand*{\arraystretch}{1.6}
\scriptsize
\begin{longtable}{|p{0.3cm}|p{14.7cm}|} 
\hline
\rowcolor[HTML]{EFEFEF} & \textbf{Question 2: What would you say a cluster consists of?}  \\
\hline
\endfirsthead
\multicolumn{2}{c}%
{\tablename\ \thetable\ -- \textit{Continued from previous page}} \\
\hline
\rowcolor[HTML]{EFEFEF} &\textbf{Question 2: What would you say a cluster consists of?}  \\
\hline
\endhead
\hline \multicolumn{2}{r}{\textit{Continued on next page}} \\
\caption{Question 2: What would you say a cluster consists of?}
\endfoot
\caption{Question 2: What would you say a cluster consists of?}
\label{w1_q2}
\endlastfoot

1 & A collection of computers unified under a single interface. By handling work loads between the individual computers a performance increase is possible.

\noindent One master computer will handle the delegation of operations to the slave units. With different strategies work load can be delegated to available units or as a predefined load per unit. Of the tasks differ in process time, the former one is preferred.
    \\ \hline

2 & A cluster consists of different computers connected between each other that forms a system, so it can be viewed as one unit. \\ \hline

3 & A Master and some and some Nodes. The Master delegates tasks. The Nodes completes tasks. \\ \hline

4 & A cluster of units working as one. \\ \hline

5 & A cluster is a collection of computers which work together in such a way that they work as one single computer. This means that a cluster are able to distribute tasks to different parts of the cluster runtime. This is different from the distributed system which functions as different computers working of different parts of a task - without any "master" to runtime distribute the tasks. \\ \hline

6 & A cluster consists of several servers which communicate and possibly keep a look out for each other. Sharing load and so forth. \\ \hline

7 & Multiple devices cooporating as one system \\ \hline

8 & Several computing units collaborates to perform defined tasks. The units communicates by local network protocols. \\ \hline

9 & It consist of multiple machines (could be virtual or physical machines) which can communicate with each other and where work loads can be distributed across the machines. \\ \hline

10 & Entry point / arbejdsfordeler

\noindent De forskellige cluster servere / processor

\noindent Config server \\ \hline
    
11 & A group of computers connected via a communication interface with work together \\ \hline

12 & A cluster as i see it usually consist of a master and some slaves. The best example i can give that we're familiar with is again the pi cluster used in the lecture. This cluster consists of a master pi that distributes tasks to the slaves. \\ \hline

13 & A cluster consists of multiple processing units and a controller. The controller manages which processing unit calculates what. \\ \hline

\multicolumn{2}{r}{\textbf{Evaluation:} Unistructural} \\ 
\end{longtable}
\normalsize

%%%% Q3
\renewcommand*{\arraystretch}{1.6}
\scriptsize
\begin{longtable}{|p{0.3cm}|p{14.7cm}|} 
\hline
\rowcolor[HTML]{EFEFEF} & \textbf{Question 3: What benefits and drawbacks do you see in a microservice architecture compared to a monolithic architecture?}  \\
\hline
\endfirsthead
\multicolumn{2}{c}%
{\tablename\ \thetable\ -- \textit{Continued from previous page}} \\
\hline
\rowcolor[HTML]{EFEFEF} &\textbf{Question 3: What benefits and drawbacks do you see in a microservice architecture compared to a monolithic architecture?}  \\
\hline
\endhead
\hline \multicolumn{2}{r}{\textit{Continued on next page}} \\
\caption{Question 3: What benefits and drawbacks do you see in a microservice architecture compared to a monolithic architecture?}
\endfoot
\caption{Question 3: What benefits and drawbacks do you see in a microservice architecture compared to a monolithic architecture?}
\label{w1_q3}
\endlastfoot

1 & With monolithic you can split a system so you only have to download the parts you need. This will also result in latency when getting new packages. In a microservice architecture everything is downloaded at once, but you may be getting something you dont need.
    \\ \hline
    
2 & A benefit is, that a small aspect of the system can be changed, without affecting the rest of the system, and/or require a recompilation/redeployment of the monolithic system. 

\noindent A benefit of the monolithic system, is that it is sometimes simpler to model mentally, and can be simpler to implement, if the purpose and functionality of the system are simple.

\noindent A drawback to the monolithic system, is that it can be to big to handle, and requires rework of the entire system, even just to change a small thing. A drawback to microservices, is that they can be harder to grasp mentally, since the "system" is several smaller systems working together in a whole. \\ \hline

3 & If a  monolithic system crashes the program can't run where if one microservice crashes the program may be able to continue - Depending on the services

\noindent With microservices it can be easier to distributed ressources \\ \hline

4 & microservices can be replaced or updated individually, with out touching the other elements in the system. \\ \hline

5 & A major benefit is the ability to increase resources at a specific part of the application without the overhead of increasing unused parts. This gives a flexibility when operating the product that is not possible with monolithic applications.

\noindent A drawback is the handling of each resource and the communications between them. In small applications this overhead is unnecessary. \\ \hline

6 & Personally i think that the microservices are much easier to maintain than a monolithic application since the idea behind microservices is that they have loose coupling and a small area of responsibility. Compared to a monolithic application where components often becomes tightly coupled which often makes it a hell to maintain.

\noindent If the application is small I believe that the microservice architecture is a bit overkill since there is a lot of preparation that must be done instead of just creating a small monolithic application. \\ \hline

7 & Better scalability because you can spin up differed amount of micro services of a specific type

\noindent More complexity due to the increased amount of interfaces.

\noindent Faster deployment cycle, because you can redeploy a single micro service

\noindent Increased flexibility in the choice of technology for a given problem.

\noindent Increased overhead.  \\ \hline

8 & Monolithic architecture, might quickly grow complex, and one change might result in a cascade of changes throughout, whereas a change in a service as long as it keeps the interface won't cause problems for others. \\ \hline

9 & Benefits: Easiere to work in parallel. - Defined interfaces. Fullfill your end of the bargain and everything works out. - If 1 server goes down should not take the application down. - Easiere to maintain and expand. - Spin up more of specific services if load is particularly heavy on the specific service

\noindent Drawbacks: Maybe more overhead on communication since its often over some ethernet connection instead of just 1 big program. - Probably overkill on small web applications. \\ \hline

10 & + Easier to maintain + Better division of labour for developer + Better scalability - Distribution systems is complex \\ \hline

11 & The drawback of a monolith general lies in the "mix of everything". Any service have to have the same conception of a "thing", no matter what context it is in. This also makes it harder to scale a system due to the hard binding between functionality. The benefits of a microservice architecture lies in the seperation of concerns. The functionality is seperated into smaller chunks and the chunks which would be under most stressed can be individually scaled up. \\ \hline

12 & Microservices (cloud computing espc.) can come with some latency when communicating over a network.Microservices are easier to expand than monolithic applications, this correlates to OPC.

\noindent Microservices might be cheaper to develop, if you're able to use already existing parts, and just add your own services to these "parts". \\ \hline

13 & Det indledende arbejde ved at opdele en applikation til mindre dele, skulle gerne på sigt betale sig da mindre applikationer er nemmere at håndtere. \\ \hline

\multicolumn{2}{r}{\textbf{Evaluation:} Unistructural} \\ 
\end{longtable}
\normalsize

















