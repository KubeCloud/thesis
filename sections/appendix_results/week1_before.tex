%!TEX root = ../../master.tex
\subsubsection*{Week 1: Reflection before course}


%%%% Q1
\renewcommand*{\arraystretch}{1.6}
\scriptsize
\begin{longtable}{|p{0.3cm}|p{14.7cm}|} 
\hline
\rowcolor[HTML]{EFEFEF} & \textbf{Question 1: What is cloud computing?}  \\
\hline
\endfirsthead
\multicolumn{2}{c}%
{\tablename\ \thetable\ -- \textit{Continued from previous page}} \\
\hline
\rowcolor[HTML]{EFEFEF} &\textbf{Question 1: What is cloud computing?}  \\
\hline
\endhead
\hline \multicolumn{2}{r}{\textit{Continued on next page}} \\
\caption{Question 1: What is cloud computing}
\endfoot
\caption{Question 1: What is cloud computing}
\label{w0_q1}
\endlastfoot

1 & Man går fra at skulle dowloade en applikation lokalt, til at man nu kan have en applikation på internettet, og dermed have den med over alt. Cloud's karakteristiker er at den har meget opbevaringsplads.  \\ \hline 

2 & Running applications or handling data remotely, from your own computer. \\ \hline 

3 & If a workload (eg server) is not hostet on a specific server but run on one or multiple web server. These servers are often not specifically build for that server application but supply computing time too multiple customers. Examples of this are Amazon AWS and Microsoft Azure.\\ \hline 

4 & Distributed computer systems hosting applications available to others thrue the internet. \\ \hline

5 & Cloud computing is communicating using a network as connection??? \\ \hline

6 & A lot of CPUs and a lot of disk space \\ \hline

7 & Cloud computing is about offloading computations to a server (or multiple servers) that is placed somewhere in the world and connected to the internet. \\ \hline

8 & Services og anden software der ligger i skyen kan eksempelvis køres vha. en browser uden installation på en lokal computer. \\ \hline

9 & when software rely on servers to deliver a service \\ \hline

10 & Delocalisering af lagerplads og/eller processorkræft. \\ \hline

11 & Cloud computing er noget, der kan tilgåes hele tiden, og hvor som helst, hvis man har internet. Det giver adgang til mange af de ting man kan på en normal computer. Eksempler på cloud computing: Dropbox, iCloud, Google drive(inkluderende Docs, sheets osv.). \\ \hline

12 & hosting your applications in the "cloud" where the cloud is a service with multiple data/server centers \\ \hline

13 & Computing happening in the "cloud", where the cloud is being a server/cluster of servers connected to the user over the internet, and "computing" being a service offered to the user, eg. file-hosting, data-processing or similar. \\ \hline

14 & I would say its like in WSN data aggregation, where we store data and compute. As we know the devices we use are getting smaller and smaller with less memory and computation power therefore its important to have some centralized unit to do stuff. \\ \hline

15 & As far as i know cloud computing is a system (and/or application) hosted in the cloud. It is coded locally, but is deployed to the cloud, so its ressources can be shared to multiple devices. \\ \hline

16 & I don't know. My guess is where you program apps which runs in "the cloud".. \\ \hline

17 & Cloud computing is about using some well-known and well-developed concepts in the cloud, to optimize the use of cloud. \\ \hline

\multicolumn{2}{r}{\textbf{Evaluation:} Prestructural} \\ 

\end{longtable}
\normalsize

%%%% Q2
\renewcommand*{\arraystretch}{1.6}
\scriptsize
\begin{longtable}{|p{0.3cm}|p{14.7cm}|} 
\hline
\rowcolor[HTML]{EFEFEF} & \textbf{Question 2: What is a distributed system}  \\
\hline
\endfirsthead
\multicolumn{2}{c}%
{\tablename\ \thetable\ -- \textit{Continued from previous page}} \\
\hline
\rowcolor[HTML]{EFEFEF} &\textbf{Question 2: What is a distributed system}  \\
\hline
\endhead
\hline \multicolumn{2}{r}{\textit{Continued on next page}} \\
\caption{Question 2: What is a distributed system}
\endfoot
\caption{Question 2: What is a distributed system}
\label{w0_q2}
\endlastfoot

1 & Distribuerede systemer fordeler arbejdet med data og prosesser rundt i stedet for at arbejde på det ét sted   \\ \hline 

2 & A distributed system is a system that isn't running on a single machine but on multiple at ones. This can be done to utilize the computing power of many computers or to access different hardware nodes. They are often managed by a command and control server. \\ \hline 

3 & Et distribueret system er et system der er fodelt udover flere maskiner såsom computere og servere. Internettet er et eksempel på et kæmpe distribueret system. \\ \hline 

4 & A distributed system is a system where multiple componentes (both soft- and hardware applications) collaborate to solve specific tasks. An example of a distributed system i Google Docs which provides the functionality of editing documents by multiple users on the same time. \\ \hline 

5 & Several devices collaborating to solve a problem \\ \hline 

6 & Et system/service fordelt på forskellige 'programstumper' \\ \hline 

7 & A distributed system is a system that consists of common parts/elements placed different in a system which in cooperation together creates the system. \\ \hline 

8 & The system in which the devices can compute on their behalf but they are all connected to each other. The devices send higher level data to each other for example a device measuring temperature and we want to know when ever the temperature is higher then desired, the device does send values of temperature constantly but only high or low. 

\noindent A distributed system is a network of computing units and they can talk and exchange data and learn from each other ( learning from each other is more pervasive computing thing but it is also valid for distributed systems).  \\ \hline 
	
9 & software communicating across multiple computers or server \\ \hline 

10 & A system based on components hosted on different physical locations, running as "one". \\ \hline 

11 & a system which has been distributed one way or another \\ \hline 

12 & Et distributed system er et system hvor man kan have mange computer som er forbundet i et netværk, og som kommunikere med hinanden f.eks. over Ethernet. \\ \hline 

13 & I think it is where you have a system which is spread out on several units/pcs, and they can communicate with each other \\ \hline  

14 & A system where the "entire system" is not necessarily located at the same place. I.e. something runs on the users computer, which then messages a server, which does some other work and returns it to the user. \\ \hline 

15 & A system which is placed at a customer??? \\ \hline 

16 & Multiple computers with a common purpose communicating with each other with out-of-process techniques. \\ \hline 

17 & I have a very limited knowledge here. But something about many computers connected together in a single system. \\ \hline 

\multicolumn{2}{r}{\textbf{Evaluation:} Unistructural} \\ 

\end{longtable}
\normalsize

%%%% Q3
\renewcommand*{\arraystretch}{1.6}
\scriptsize
\begin{longtable}{|p{0.3cm}|p{14.7cm}|} 
\hline
\rowcolor[HTML]{EFEFEF} & \textbf{Question 3: What is cluster computing?}  \\
\hline
\endfirsthead
\multicolumn{2}{c}%
{\tablename\ \thetable\ -- \textit{Continued from previous page}} \\
\hline
\rowcolor[HTML]{EFEFEF} &\textbf{Question 3: What is cluster computing?}  \\
\hline
\endhead
\hline \multicolumn{2}{r}{\textit{Continued on next page}} \\
\caption{Question 3: What is cluster computing?}
\endfoot
\caption{Question 3: What is cluster computing?}
\label{w0_q3}
\endlastfoot

1 & Multiple computers working together to solve the same problem?   \\ \hline

2 & when multiple computers/servers are seen as one computer/server. \\ \hline

3 & maybe a lot of small applications dependant on one another and thereby forming a cluster? \\ \hline

4 & A system where the processing is distributed over several workstations/computers. Thereby reducing the workload on each node, and decreasing the time needed to complete the task. \\ \hline

5 & Cluster computing is when multiple servers are connected to each other (forming a cluster). An example where clustering is useful is when working with databases where clusters can be used to store data on several machines to ensure that no data is lost even though one server is down. \\ \hline

6 & Without googling it ! I have no clue 

\noindent I would say cluster computing could be where we program a number of devices where they share some memory and they can talk to each other.  \\ \hline

7 & Dont know \\ \hline

8 & A computer cluster is an array of computers with are interconnected to form a single fast computer. \\ \hline

9 & Et andet ord for cloud service / distributed system set mere teknisk. 

\noindent En gruppe servere der (tilfældigt) bliver tildelt en arbejdsopgave.   \\ \hline

10 & - \\ \hline

11 & Cluster computing is when computing is gathered in groups and each group has a responsibility for calculating a certain area of the system. \\ \hline

12 & Cluster er en gruppe af computer som er linket sammen og kan interagere med hinande. Et example kunne være et datacenter.\\ \hline

13 & I don't know. It sounds like combining the hardware performance of a cluster og units to achieve one cluster capable of more than the single unit. \\ \hline

14 & Multiple computers connected, so that they create one computer. \\ \hline

15 & A computer setup where several computers aim to solve the same task. This increases the ability to handle large numbers of connection or solve tasks faster. \\ \hline

16 & Computere der arbejder sammen om at løse opgaver. Hver computer har en funktionalitet \\ \hline

\multicolumn{2}{r}{\textbf{Evaluation:} Unistructural} \\ 

\end{longtable}
\normalsize


%%%% Q4
\renewcommand*{\arraystretch}{1.6}
\scriptsize
\begin{longtable}{|p{0.3cm}|p{14.7cm}|} 
\hline
\rowcolor[HTML]{EFEFEF} & \textbf{Question 4: What is virtualization?}  \\
\hline
\endfirsthead
\multicolumn{2}{c}%
{\tablename\ \thetable\ -- \textit{Continued from previous page}} \\
\hline
\rowcolor[HTML]{EFEFEF} &\textbf{Question 4: What is virtualization?}  \\
\hline
\endhead
\hline \multicolumn{2}{r}{\textit{Continued on next page}} \\
\caption{Question 4: What is virtualization?}
\endfoot
\caption{Question 4: What is virtualization?}
\label{w0_q4}
\endlastfoot

1 & multiple servers running on the same hardware. multiple OS's on one computer.    \\ \hline

2 & Virtualization is when a system runs inside another system. \\ \hline

3 & Virtualization is used to improve the use of physical resources. A single physical server might for example host multiple virtual servers with are running independent OS as have access to a subset of the physical servers resources.\\ \hline

4 & Running a computer inside a computer. Allowing for multiple instances running on the same hardware at once.\\ \hline

5 & To virtualize is that you can simulate something. My experience is limited to virtual machines, where a operating system is virtualized.\\ \hline

6 & An abstraction of the physical machines running the code.\\ \hline

7 & Creating a not real but virtual version of somethings such as devices or could be a server or some kind of storage devices or a whole operating system.\\ \hline 

8 &  hmm i dont know :(\\ \hline

9 & .\\ \hline

10 & Making a digital rendition of something physical (or other digital applications), within a sandbox on a digital system.\\ \hline

11 & Fx et level of indirection i mellem det software der afvikles og den hardware det afvikles på. 

\noindent NT kernens memory management er et eksempel herpå, de forskellige programmer benytter sig af virtuelle og ikke fysiske adresser. \\ \hline

12 & Virtualization is about abstracting away hardware specific implementations. An example could be a virtual machine which can be used to run test environments etc. Another example could be virtual memory (I believe it's called that) which is used if a computer runs out of RAM and instead allocates temporary space on the HDD.\\ \hline

13 & I don't know \\ \hline

14 & Making something virtually - meaning not physically, rather something only existing virtually \\ \hline

\multicolumn{2}{r}{\textbf{Evaluation:} Unistructural} \\ 

\end{longtable}
\normalsize


%%%% Q5
\renewcommand*{\arraystretch}{1.6}
\scriptsize
\begin{longtable}{|p{0.3cm}|p{14.7cm}|} 
\hline
\rowcolor[HTML]{EFEFEF} & \textbf{Question 5: Describe some benefits and liabilities of sequential vs parallel execution}  \\
\hline
\endfirsthead
\multicolumn{2}{c}%
{\tablename\ \thetable\ -- \textit{Continued from previous page}} \\
\hline
\rowcolor[HTML]{EFEFEF} &\textbf{Question 5: Describe some benefits and liabilities of sequential vs parallel execution}  \\
\hline
\endhead
\hline \multicolumn{2}{r}{\textit{Continued on next page}} \\
\caption{Question 5: Describe some benefits and liabilities of sequential vs parallel execution}
\endfoot
\caption{Question 5: Describe some benefits and liabilities of sequential vs parallel execution}
\label{w0_q5}
\endlastfoot

1 & with parallel execution can use multiple cpu cores, but tasks are not executed in a specific order, where sequential is using only one cpu core but gives an order of execution.     \\ \hline

2 & When using sequential executions the order of execution is ensured and is beneficial if we want to perform tasks in a specific order. A disadvantage of using sequential execution is that small tasks may be delayed by tasks taking long time.

\noindent Parallel execution doesn't preserve the order of tasks which means that synchronization can be hard to do (mutual exclusion etc.). But tasks doen't block each other which means that small tasks aren't delayed as much by large tasks as in sequential execution. \\ \hline

3 & The advantage of parallel execution is that a specific workload can be completed faster by calculation it on multiple cpu/gpu's at ones. The disadvantage is that the pieces of work are finished in an unknown order. \\ \hline

4 & Benefits of sequential execution: - only one calculation at a time, (maybe more precise), - time can easier be calculated

\noindent Liabilities of sequential execution: - slower calculation than parallel

\noindent Benefits of parallel execution: - faster calculation, - multiple events at once

\noindent Liabilities of sequential execution: - harder to calculate time(delay) \\ \hline

5 & A benefit of sequential processing is the path of execution is always known. A downside is that only one piece of code can run at a time, which is a liability when waiting for external events/polling.

\noindent A benefit of parallel execution is that multiple pieces of code can be run simultaneously, which prevents code from blocking in certain circumstances. A liability is that the order of execution is not always known, since the scheduler manages which code runs when. \\ \hline

6 & Sequential execution: Predictable sequence of execution. Some what possible to expect a certain processing time. Easy to get right.

\noindent Parallel exection: Non-predictable sequence of execution. Often shorter processing time if the individual tasks to be accomplished are independent of each other. Harder to get right. \\ \hline

7 & I sequential execution fortager man kun en ting ad gangen, og dette kan resultere i stor latency. Mens man i parallel execution foretager flere ting parllelt, og derved kan latency nedbringes. \\ \hline

8 & Well it depends on what the application is to be used for. The seq execution is good when there is no need of threads and there is no shared memory and the program has only few task's to take care of. If there are more then one task then to minimize the delay and to provide some better availability data it is important to have parallel executions of applications. 

\noindent The draw back regarding parallel is that there is a need of having some kind of operating system on the device which enable threading. (FREEARTOS could be a valid candidate) \\ \hline

9 & Ved sekventiel eksekvering sikre man sig at opgaverne bliver udført i den ønskede rækkefølge. Dog tager dette lige så lang tid som summen af alle opgaverne.

\noindent Ved parrallel eksekvering kan flere opgaver løses på samme tid. Dog kan man komme ud for at der skal koordineres imellem forskellige prossesor. Ydermere kan prossesor vente på hinanden og værste tilfælde skabe et deadlock. Der kan spares meget til ved benyttelse af parrallel eksevering \\ \hline

10 & sequential execution has the benefit of being thread safe and if you have one task that depends on another, you put them in the correct order and run them sequential so that, "that" one task will have what it needs because the previous task delivered it

\noindent parallel execution is faster since it performs several tasks at once, but is more complicated to setup and harder to get right \\ \hline

11 & Sekventielt er mere sikkert da man hele tiden har styr på, hvornår hvilke ting de sker, kan dog være langsommere en parralel.

\noindent Parallel er hurtigere, dog skal der tages højde for nogle ting som, f.eks. deadlocks, og at, hvis man lader flere tråder tilgå det samme data på samme tid, kan det ende galt. \\ \hline

12 & Benefits: Sequential is easy, Parallel is faster

\noindent Liabilities: Parallel is complex, Sequential is slow \\ \hline

13 & Sekventiel: + Ofte simpler, billgere. - Det hele falder sammen hvis ét led bryder sammen. 

\noindent Parallelt: + Kan nemt skaleres. + Problemer ét sted har ikke nødvendigvis indflydelse på systemet som heldhed. - Mere kompliseret, større anskaffelses omkostninger. - Visse arbejdsopgaver egner sig dårligt til parallel afvikling. \\ \hline

14 & Benefits of sequential execution can be order of execution. You know how far and how long you are all the time. Easier debugging. 

\noindent Liability of sequential execution can be the delay of actions. This is especially noticable on a GUI.

\noindent The benefits and liabilites of parallel execution is the opposite of what I just mentioned. \\ \hline

15 & The problems of sequential execution is the risk of bottlenecking. We cannot execute more than one command at one time and mutiple commands will stack up in a queue if one or more take longer to execute. The problem of parallel execution is not knowing the execution order. Any command might be executed in any order. \\ \hline

16 & Parallel execution is usually faster than sequential, however, sequential execution ensures that no deadlocks/race conditions occur.

\noindent Parallel execution offers multitasking, both in the UI and logical execution of programs. \\ \hline

17 & With sequential execution you can handle when and what (for example) code will be executed. It is a benefit that you know which order the execution will be done.

\noindent The opposite parallel execution is a benefit that you can run multiple tasks at once.

\noindent The benefit from sequential is also the weakness of parallel execution and vice versa. \\ \hline

\multicolumn{2}{r}{\textbf{Evaluation:} Unistructural} \\ 

\end{longtable}
\normalsize


%%%% Q6
\renewcommand*{\arraystretch}{1.6}
\scriptsize
\begin{longtable}{|p{0.3cm}|p{14.7cm}|} 
\hline
\rowcolor[HTML]{EFEFEF} & \textbf{Question 6: Describe your understanding of possible errors in communicating over a network}  \\
\hline
\endfirsthead
\multicolumn{2}{c}%
{\tablename\ \thetable\ -- \textit{Continued from previous page}} \\
\hline
\rowcolor[HTML]{EFEFEF} &\textbf{Question 6: Describe your understanding of possible errors in communicating over a network}  \\
\hline
\endhead
\hline \multicolumn{2}{r}{\textit{Continued on next page}} \\
\caption{Question 6: Describe your understanding of possible errors in communicating over a network}
\endfoot
\caption{Question 6: Describe your understanding of possible errors in communicating over a network}
\label{w0_q6}
\endlastfoot

1 & Main scenario is loss of connection or a destination not responding / responding with wrong information.

\noindent No compile time check on the external resources. If you don't understand the API your are talking to, you don't get a heads
    \\ \hline

2 & There can be error in communicating over networks on many levels. The cloud can be causing errors, as well as link layer, computers, software. One important error is that data multiple users handle at the same time should be locked so no two users write at the same time. \\ \hline 

3 & packet loss, connection loss, race conditions. \\ \hline 

4 & Data kan blive corruptet og derved, er det en god ide at encode og decode ens data.  \\ \hline 

5 & Possible errors with network communication are: - no time synchronization, - lost messages, - wrong format, - bit errors \\ \hline 

6 & Communication interfaces need to be up to date. If one system updates their interface without the others knowing. Things can go wrong.

\noindent Bit errors if using some old communication protocol i guess?

\noindent Services that fail or respond slowly. Maybe because of heavy load. \\ \hline 

7 & The main issues, as i see it, are packet loss or data corruption/noise.

\noindent There is also the matter of network com. being somewhat time consuming. \\ \hline 

8 & Possible errors are loss of packets, loss of connection, accidental or malicious corruption of data, change in order of packets, \\ \hline 

9 & Kommunikationen kan ikke gå hurtigere end den hastighed som netværket tilbyder. Bottlenecks kan derfor være et problem.

\noindent At flytte storer mængder data over et netværk kan være en langsommelig proces.

\noindent Mistes der forbindelse over netværket kan prosseser gå i står eller tabe data \\ \hline 

10 & Possible errors might be a lost connection, timeout, error in transmission, lost messages etc. This requires the system to resend the messages (when it has discovered that they were lost), imposing a delay in the execution. \\ \hline 

11 & Network communication is often done using some sort of packages or buffers filled with data. This means that data have a lot of possible places to go wrong. This can be due to packet loss, missed packages, wrong  order of packages might mean misinterpreted packages, package corruption, package interception and so on. \\ \hline 

12 & Et problem med kommunikation over netværk er tab af pakke, der gør at en potentiel besked går tabt, hvis den er lille nok, eller bare ender som garbage fordi der mangler pakker. \\ \hline 

13 & Whenever sending and receiving information across a network, the information can be sniffed by someone unauthorized. Loss of information packages can occur.  \\ \hline 

14 & When communicating over a network a lot of things can go wrong. Some errors which can occur is the following: Package loss, Server/component is down (can't be requested), Network cables can be unplugged or destroyed in some way \\ \hline 

15 & Packet loss, Packet received with errors, Error under decoding the message, Packet delay, CSMA/CD in which the delay of receiving data can be random due to 2 nodes try to send data at the same time and collide. \\ \hline 

16 & Packet loss - information når ikke frem. Dårligt forbindelse / støj, den information der kommer frem er måske korrupt. Timing errors, de kommunikerende stykker hardware er uenige om hvordan 0 og 1 sendes, eller er ikke gode nok til at holde sig til aftalte regler. Pakke nummer 7 når frem før pakke nummer 3. En eller anden spade har sat to DHCP servere på samme subnet. \\ \hline 

\multicolumn{2}{r}{\textbf{Evaluation:} Unistructural} \\ 

\end{longtable}
\normalsize



%%%% Q7
\renewcommand*{\arraystretch}{1.6}
\scriptsize
\begin{longtable}{|p{0.3cm}|p{14.7cm}|} 
\hline
\rowcolor[HTML]{EFEFEF} & \textbf{Question 7: Describe your understanding of resilience in a system}  \\
\hline
\endfirsthead
\multicolumn{2}{c}%
{\tablename\ \thetable\ -- \textit{Continued from previous page}} \\
\hline
\rowcolor[HTML]{EFEFEF} &\textbf{Question 7: Describe your understanding of resilience in a system}  \\
\hline
\endhead
\hline \multicolumn{2}{r}{\textit{Continued on next page}} \\
\caption{Question 7: Describe your understanding of resilience in a system}
\endfoot
\caption{Question 7: Describe your understanding of resilience in a system}
\label{w0_q7}
\endlastfoot

1 & The ability of a system to recover from an error or try to handle errors in a proper way.
    \\ \hline
    
2 & Ved ikke hvad der menes. \\ \hline

3 & The ability to change. High resilience will be easy to maintain and vice versa. \\ \hline

4 & The resilience of a system is its ability to detect and recover from errors. \\ \hline

5 & De forholdsregler der er taget for at sikre at systemet ikke fejler. Dette kan være lige fra strømafbrydelse til et evigt while-loop. Jo mere modstandsdygtig et system er lavet desto lavere er sandsynligheden for fejl \\ \hline

6 & Robusthed i et system er ikke nødvendigvis at der ikke sker fejl i systemet, det kan nærmest ikke undgåes at der sker fejl. Det er mere at systemet kan reagere på de fejl, og håndtere dem. Et eksempel kunne være et system, hvor de individulle dele når de kommunikerer, kan melde om pakketab og sende et request ud efter den manglende pakke, så der rettes op på fejlen. \\ \hline

7 & a resilient system is one which is almost impossible to bring to its knees. You can maybe disable one of its services and it will still run. You can run excessive load on it and it will still be responsive and not break down. \\ \hline

8 & The ability to handle failures in every aspect of the system gracefully and recover automatically. \\ \hline

9 & Resillence in a system is its ability to withstand the problems which is known in a system. A very resillent system is good to withstand both expected and unexpected problems. \\ \hline

10 & I am assuming that resilience has got something to do with handling incoming errors/corrupted data, and being able to defend itself from malware and viruses. \\ \hline

11 & Evnen til at håndtere ureglmæssigheder. Hvilket igen kan forstås på flere måder.

\noindent- Prøv igen x antal gange.

\noindent - Informér om det.

\noindent - Forsøg at mindske følgerne af system/service fejl så meget som muligt. (Lad vær med at slukke for moteren på en bil hvis vinduerne ikke vil rulle ned) \\ \hline

12 & No clue \\ \hline

13 & I don't know. \\ \hline

14 & a system staying functional even when some servers crash or loose connection. \\ \hline

15 & Some sort of protection against unauthorized access.\\ \hline

16 & Resilience is how good the system is to manage errors occuring during execution. The better the system handles errors, the more resilient it is.\\ \hline

\multicolumn{2}{r}{\textbf{Evaluation:} Unistructural} \\ 

\end{longtable}
\normalsize
