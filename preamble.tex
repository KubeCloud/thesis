%!TEX root = master.tex
%---------------------------------------------------------------------------%
%---------------------------- PACKAGE CONTROL ------------------------------%
%---------------------------------------------------------------------------%
\usepackage[utf8]{inputenc} % Input encoding - Depending on editor
\usepackage[english]{babel} % Language package
\usepackage{fix-cm} % Fix for cm
\usepackage[]{graphicx} % To handle pictures
\usepackage[table,xcdraw, dvipsnames]{xcolor}
\usepackage{tikz} % Graphical tool
\usepackage{mathtools} % To use \eqref
\usepackage{url} % Use of urls in the text
\usepackage{sansmath,subfig} % Gives a warning because subfig loads caption
\usepackage{datetime}
\usepackage{pdfpages}
\usepackage{booktabs}
\usepackage{float}
%\usepackage{epstopdf}
\usepackage{caption}
\usepackage[bottom]{footmisc}
\usepackage{microtype} % Slightly tweak font spacing for aesthetics
\usepackage{listings}
\usepackage[plainpages=false,pdfpagelabels,pageanchor=false]{hyperref} % Active links in PDF
\usepackage[sfdefault,light,condensed]{roboto}  %% Font Roboto
\usepackage[T1]{fontenc}
\usepackage{amsmath,amssymb,amsthm} % For including math equations, theorems, symbols, etc
\usepackage{longtable}
\usepackage{amsmath}
\usepackage{mwe}
\usepackage{graphbox} %loads graphicx package
\usepackage{paralist}
\usepackage{subfig}
\expandafter\def\expandafter\UrlBreaks\expandafter{\UrlBreaks%  save the current one
  \do\a\do\b\do\c\do\d\do\e\do\f\do\g\do\h\do\i\do\j%
  \do\k\do\l\do\m\do\n\do\o\do\p\do\q\do\r\do\s\do\t%
  \do\u\do\v\do\w\do\x\do\y\do\z\do\A\do\B\do\C\do\D%
  \do\E\do\F\do\G\do\H\do\I\do\J\do\K\do\L\do\M\do\N%
  \do\O\do\P\do\Q\do\R\do\S\do\T\do\U\do\V\do\W\do\X%
  \do\Y\do\Z}
  
%---------------------------------------------------------------------------%
%---------------------------- GRAPHICS CONTROL -----------------------------%
%---------------------------------------------------------------------------%
% Path for graphics
\graphicspath{{figures/}} % Image path


%---------------------------------------------------------------------------%
%---------------------------- COLOR CONTROL --------------------------------%
%---------------------------------------------------------------------------%
\definecolor{ase_blue}{RGB}{46,158,217}
%\definecolor{ase_blue}{RGB}{10,55,136} % The ASE blue color
\definecolor{ocre}{RGB}{243,102,25}
\definecolor{wetAsphalt}{RGB}{52, 73, 94}
\definecolor{cloudsColor}{RGB}{236, 240, 241}
\definecolor{emerald}{RGB}{52, 152, 219}
\definecolor{carrot}{RGB}{230, 126, 34}
\definecolor{alizarin}{RGB}{231, 76, 60}
\definecolor{emerald_green}{RGB}{46, 204, 113}
\definecolor{darkblue}{HTML}{000080}
\definecolor{darkgreen}{RGB}{0, 136, 0}
\definecolor{funnygreen}{RGB}{0, 102, 102}
\definecolor{airforceblue}{rgb}{0.36, 0.54, 0.66}
\definecolor{mercury}{RGB}{218, 218, 218}

\colorlet{punct}{red!60!black}
\definecolor{background}{HTML}{EEEEEE}
\definecolor{delim}{RGB}{20,105,176}
\colorlet{numb}{magenta!60!black}


%---------------------------------------------------------------------------%
%---------------------------- MARGIN CONTROL -------------------------------%
%---------------------------------------------------------------------------%
\setlrmarginsandblock{3cm}{2.5cm}{*}
\setulmarginsandblock{3cm}{*}{1.2}
\checkandfixthelayout[nearest]
\setlength{\evensidemargin}{\oddsidemargin}


%--------------------------------------------------------------------------%
%------------------------- PAGESTYLE - PROPERTIES -------------------------%
%--------------------------------------------------------------------------%
%\renewcommand{\chaptermark}[1]{\markboth{\MakeUppercase{#1}}{}}

\makepagestyle{ase_report}
\makeoddhead{ase_report}{}{\small\sffamily\leftmark}{}
\makeoddfoot{ase_report}{}{}{\small\sffamily\thepage}

\makeatletter
\makepsmarks{ase_report}{%
  \renewcommand\chaptermark[1]{%
    \markboth{%
      \ifnum \value{secnumdepth} > 1
      \if@mainmatter % 
      \@chapapp\ \thechapter. \ %
      \fi
      \fi
      ##1}{}}%
  \renewcommand\tocmark{\markboth{\contentsname}{\contentsname}}%
  \renewcommand\lofmark{\markboth{\listfigurename}{\listfigurename}}%
  \renewcommand\lotmark{\markboth{\listtablename}{\listtablename}}%
  \renewcommand\bibmark{\markboth{\bibname}{\bibname}}%
  \renewcommand\indexmark{\markboth{\indexname}{\indexname}}%
  \renewcommand\sectionmark[1]{\markright{##1}}%
  \renewcommand\subsectionmark[1]{\markright{##1}}%
  \renewcommand\subsubsectionmark[1]{\markright{##1}}%
}

\copypagestyle{plain}{ase_report}
\makeoddhead{plain}{}{}{}
\makeoddfoot{plain}{}{}{\small\sffamily\thepage}

\pagestyle{ase_report}
\aliaspagestyle{chapter}{plain}

%--------------------------------------------------------------------------%
%--------------------- HEADING - SECTION ----------------------------------%
%--------------------------------------------------------------------------%
\newcommand{\ruledsec}[1]{%
  \Large\bfseries\sffamily\raggedright #1
  \color{ase_blue}\rule[1pt]{\textwidth}{1.0pt}} % Section with ruler
\setsecheadstyle{\ruledsec} % Define section head style

\setfloatlocations{figure}{htp}
\setfloatlocations{table}{htp}

%--------------------------------------------------------------------------%
%--------------------- HEADING - SUBs-SECTION -----------------------------%
%--------------------------------------------------------------------------%
\addtocounter{secnumdepth}{2} % Depth numbering

\setsubsecheadstyle{\Large\bfseries\sffamily\raggedright}
\setsubsubsecheadstyle{\large\bfseries\sffamily\raggedright}

\setsechook{\hangsecnum} % Hang the section number in margin
\setsubsechook{\defaultsecnum} % Don't do this on the subsections
\setsubsubsechook{\defaultsecnum}
\setaftersecskip{5pt} % Default skip between the section and text

%--------------------------------------------------------------------------%
%------------------------- TOC - PROPERTIES -------------------------------%
%--------------------------------------------------------------------------%
\raggedbottomsectiontrue % The page may not be strected on page breaks
\setsecnumdepth{subsubsection} % Set section depth in the TOC
\maxsecnumdepth{subsubsection} % Max of section depth in the TOC
\settocdepth{subsection} % Up to and including subsection

\setlength{\cftbeforechapterskip}{1.0em plus 0.1em minus 0.1em} % Space from chapters
%\chapterprecistoc{Text in TOC}

\addto\captionsenglish{
  \renewcommand*{\cftchaptername}{Chapter{\space}}
  \renewcommand*{\cftfigurename}{Fig.{\space}}
  \renewcommand*{\contentsname}{Table of Contents}
  \renewcommand*{\abstractname}{Abstract}
  \renewcommand*{\listfigurename}{List{\space}of{\space}Figures}
  \renewcommand*{\listtablename}{List{\space}of{\space}Tables}
  \renewcommand*{\appendixtocname}{Appendices}
  \renewcommand*{\appendixpagename}{Appendices}
}

\addto\captionsdanish{
  \renewcommand*{\cftchaptername}{\space}
  \renewcommand*{\cftfigurename}{Fig.\space}
  \renewcommand*{\abstractname}{Resumé}
  \renewcommand*{\contentsname}{Indholdsfortegnelse}
  \renewcommand*{\listfigurename}{Liste{\space}af{\space}Figurer}
  \renewcommand*{\listtablename}{Liste{\space}af{\space}Tabeller}
  \renewcommand*{\appendixtocname}{Appendikser}
  \renewcommand*{\appendixpagename}{Appendikser}
}
%--------------------------------------------------------------------------%
%------------------------- CHAPTER STYLE ----------------------------------%
%--------------------------------------------------------------------------%
\makechapterstyle{ase_chapterstyle}{
  \setlength{\beforechapskip}{15pt}
  \setlength{\afterchapskip}{0.5cm}
  \renewcommand*{\printchaptername}{}
  \renewcommand*{\chapnumfont}{\normalfont\sffamily\bfseries\fontsize{60}{0}\selectfont}
  \renewcommand*{\printchapternum}{
    \flushright
    \begin{tikzpicture}
      \draw[fill,color=ase_blue] (0,0) rectangle (2cm,2cm);
      \draw[color=white] (1cm,1cm) node { \chapnumfont\thechapter };
    \end{tikzpicture}
  }
  \renewcommand*{\chaptitlefont}{\normalfont\sffamily\Huge\bfseries\color{black}}
  \renewcommand*{\printchaptertitle}[1]{%
    \raggedright\chaptitlefont\parbox[t]{\textwidth}{\raggedright##1}}
}

\chapterstyle{ase_chapterstyle}

%----------------------------------------------------------------------------------------
% THEOREM STYLES
%----------------------------------------------------------------------------------------

\newtheoremstyle{ocrenumbox}% % Theorem style name
{0pt}% Space above
{0pt}% Space below
{\itshape}% % Body font
{-0.50em}% Indent amount
{\the\thm@notefont\sffamily\noindent\color{carrot}}% % Theorem head font
{\;}% Punctuation after theorem head
{0.0em}% Space after theorem head
{\noindent\thmnote{\noindent\nobreakspace\the\thm@notefont\sffamily\bfseries\color{carrot}\nobreakspace\noindent#3 \\}} % Optional theorem note


\newtheoremstyle{wetAsphaltCloud}% QUETO
{5pt}% Space above
{5pt}% Space below
{\itshape}% Body font
{0.0em} % Indent amount
{\selectfont}% Theorem head font
{\;}% Punctuation after theorem head
{0.0em}% Space after theorem head
{\thmnote{\nobreakspace\the\thm@notefont\sffamily\bfseries\itshape\nobreakspace\noindent#3 - }}% Optional theorem note

\newtheoremstyle{blacknumex} % DEFINITION
{0pt}% Space above
{0pt}% Space below
{\selectfont}% Body font
{}% Indent amount
{\selectfont}% Theorem head font
{}% Punctuation after theorem head
{0.25em}% Space after theorem head
{\noindent\thmnote{\nobreakspace\the\thm@notefont\sffamily\bfseries\nobreakspace#3}}% Optional theorem note

\newtheoremstyle{exerciseBox}% % Theorem style name
{0pt}% Space above
{0pt}% Space below
{\selectfont}% % Body font
{-0.50em}% Indent amount
{\the\thm@notefont\sffamily\noindent\color{ase_blue}}% % Theorem head font
{\;}% Punctuation after theorem head
{0.0em}% Space after theorem head
{\noindent\thmnote{\noindent\nobreakspace\the\thm@notefont\sffamily\bfseries\color{ase_blue}\nobreakspace\noindent#3 \\}} % Optional theorem note


% Defines the theorem text style for each type of theorem to one of the three styles above
\newcounter{dummy} 
\numberwithin{dummy}{section}
\theoremstyle{ocrenumbox}
\newtheorem{theoremeT}[dummy]{Theorem}
\theoremstyle{blacknumex}
\newtheorem{definitionT}{Definition}[section]
\theoremstyle{wetAsphaltCloud}
\newtheorem{corollaryT}[dummy]{Corollary}

\theoremstyle{exerciseBox}
\newtheorem{exerciseT}[dummy]{Exercise}

%----------------------------------------------------------------------------------------
% DEFINITION OF COLORED BOXES
%----------------------------------------------------------------------------------------

\RequirePackage[framemethod=default]{mdframed} % Required for creating the theorem, definition, exercise and corollary boxes

% Theorem box
\newmdenv[skipabove=7pt,
skipbelow=15pt,
linecolor=ase_blue,
linewidth=2pt,
innerleftmargin=12pt,
innerrightmargin=12pt,
innertopmargin=12pt,
innerbottommargin=12pt,
leftmargin=0cm,
backgroundcolor=cloudsColor,
rightmargin=0cm]{tBox}


% Corollary
\newmdenv[skipabove=7pt,
skipbelow=7pt,
rightline=false,
leftline=true,
topline=false,
bottomline=false,
linecolor=emerald,
innerleftmargin=5pt,
innerrightmargin=5pt,
innertopmargin=0pt,
leftmargin=0cm,
rightmargin=0cm,
linewidth=4pt,
innerbottommargin=0pt]{dBox}  

% Definition
\newmdenv[skipabove=7pt,
skipbelow=7pt,
rightline=false,
leftline=true,
topline=false,
bottomline=false,
linecolor=wetAsphalt,
backgroundcolor=cloudsColor,
innerleftmargin=5pt,
innerrightmargin=5pt,
innertopmargin=5pt,
leftmargin=0cm,
rightmargin=0cm,
linewidth=4pt,
innerbottommargin=5pt]{cBox}  

% Exercisebox box
\newmdenv[skipabove=7pt,
skipbelow=7pt,
backgroundcolor=cloudsColor,
linecolor=ase_blue,
innerleftmargin=5pt,
innerrightmargin=5pt,
innertopmargin=5pt,
leftmargin=0cm,
rightmargin=0cm,
innerbottommargin=5pt]{eBox}   


      

% Creates an environment for each type of theorem and assigns it a theorem text style from the "Theorem Styles" section above and a colored box from above
\newenvironment{theorem}{\begin{samepage}\begin{tBox}\begin{theoremeT}}{\end{theoremeT}\end{tBox}\end{samepage}}        
\newenvironment{definition}{\begin{samepage}\begin{cBox}\begin{definitionT}}{\end{definitionT}\end{cBox}\end{samepage}}    
\newenvironment{citat}{\begin{samepage}\begin{dBox}\begin{corollaryT}}{\end{corollaryT}\end{dBox}\end{samepage}} 
\newenvironment{kasse}{\begin{samepage}\begin{eBox}\begin{exerciseT}}{\end{exerciseT}\end{eBox}\end{samepage}}   




\definecolor{mygreen}{rgb}{0,0.6,0}
\definecolor{mygray}{rgb}{0.5,0.5,0.5}
\definecolor{mymauve}{rgb}{0.58,0,0.82}

\lstset{ %
  backgroundcolor=\color{cloudsColor},   % choose the background color; you must add \usepackage{color} or \usepackage{xcolor}
  basicstyle=\footnotesize\ttfamily,        % the size of the fonts that are used for the code
  breakatwhitespace=false,         % sets if automatic breaks should only happen at whitespace
  breaklines=true,                 % sets automatic line breaking
  captionpos=b,                    % sets the caption-position to bottom
  commentstyle=\color{mygreen},    % comment style
  deletekeywords={...},            % if you want to delete keywords from the given language
  escapeinside={\%*}{*)},          % if you want to add LaTeX within your code
  extendedchars=true,              % lets you use non-ASCII characters; for 8-bits encodings only, does not work with UTF-8
  frame=single,                    % adds a frame around the code
  keepspaces=true,                 % keeps spaces in text, useful for keeping indentation of code (possibly needs columns=flexible)
  keywordstyle=\color{blue},       % keyword style
  language=Python,                 % the language of the code
  otherkeywords={*,...},            % if you want to add more keywords to the set
  numbers=left,                    % where to put the line-numbers; possible values are (none, left, right)
  numbersep=5pt,                   % how far the line-numbers are from the code
  numberstyle=\tiny\color{mygray}, % the style that is used for the line-numbers
  rulecolor=\color{black},         % if not set, the frame-color may be changed on line-breaks within not-black text (e.g. comments (green here))
  showspaces=false,                % show spaces everywhere adding particular underscores; it overrides 'showstringspaces'
  showstringspaces=false,          % underline spaces within strings only
  showtabs=false,                  % show tabs within strings adding particular underscores
  %stepnumber=2,                    % the step between two line-numbers. If it's 1, each line will be numbered
  stringstyle=\color{mymauve},     % string literal style
  tabsize=2,                     % sets default tabsize to 2 spaces
  title=\lstname                   % show the filename of files included with \lstinputlisting; also try caption instead of title
}
%%%%%%%%%%%%%%%%%%%%%%%%%% OTHER FIXES
\DeclareUnicodeCharacter{00A0}{ }



%%%%% INVOICE

\usepackage{fp}
%\usepackage{ragged2e}
\newcounter{cnt}
\setcounter{cnt}{0}
\def\inc{\stepcounter{cnt}\thecnt}
\gdef\TotalHT{0}
\newcommand{\product}[3]{%
\inc &#1  &#2   &#3  &\FPmul\temp{#2}{#3}\FPround\temp{\temp}{2}\temp 
%% Totalize
\FPadd\total{\TotalHT}{\temp}%
\FPround\total{\total}{2}%
\global\let\TotalHT\total%
\\ }
\newcommand{\totalttc}{
   \TotalHT  }